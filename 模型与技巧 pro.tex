\documentclass{ctexbook}
\pagestyle{plain}
\makeatletter
\renewcommand{\cleardoublepage}{\clearpage}
\makeatother
\usepackage{graphicx}
\graphicspath{{photo/}}
\usepackage{float}
\usepackage{amsmath,amssymb}
\usepackage{geometry}
\geometry{a4paper, margin=1in}
\usepackage{hyperref}
\hypersetup
{
  hidelinks,
}
\usepackage{enumerate}
\usepackage{newunicodechar}
\newunicodechar{、}{,}
\newunicodechar{①}{\textcircled{1}}  
\newunicodechar{②}{\textcircled{2}}  

\title{模型与技巧}
\author{Logic\\ZnRa}
\date{\today}

\begin{document}
\maketitle

\chapter*{序言}
本书的主要目的在于帮助理解题目背后隐藏的原理以及解题思路,花费了学长们很多的心力——主要是Logic学长的心力。

数学是工具,现代社会离不开数学。因此学好数学很重要,不仅仅是为了高考,还对于大学中的高等数学的理解有着很大用处。当今社会对于数学越来越重视,这一点很好地反映在高考中。

根据2025年的数学高考卷来看,大家对于知识点背后的原理需要有着透彻的认识,这样才能做到临危不乱。知道了原理,熟练运用,那么无论题目的再怎么创新,心里都会有大致的思路,内心才不会畏惧题目。高考数学对于选拔性质的加强体现在它对于题目的分流处理,简单的题目极其简单,困难的题目极其困难,这就需要学弟们自己去钻研题目、理解知识、熟络教材。这些都是需要自己驱动的,“纸上得来终觉浅,绝知此事要躬行”。我们只是帮助者,真的成绩还是需要自己奋斗。

送给各位高三学弟一句名言来勉励大家,希望大家能在高三的最后岁月里不留遗憾。“对未来的真正慷慨是把一切都献给现在——加缪”。

最后,欢迎各位学弟报考杭州电子科技大学!——by ZnRa

\vspace{1cm}
\chapter*{学长寄语}
\noindent Logic:
\begin{center}
有志者,事竟成,破釜沉舟,百二秦关都属楚。\\
苦心人,天不负,卧薪尝胆,三千越甲可吞吴。
\end{center}
\noindent DCAG:
\begin{center}
We are all in the gutter,but some of us are looking at the stars.
\end{center}
\noindent NITORI:
\begin{center}
幻梦终醒,本无不散之宴,却不悔付此华年
\end{center}
\noindent WOOD:
\begin{center}
All tragedy erased.I see only wonders.
\end{center}
\noindent YOGURT:
\begin{center}
桐高凤必至,花香蝶自来
\end{center}
\noindent SHEEPZTHER:
\begin{center}
You don't need a weatherman to know which way the wind blows.
\end{center}
\noindent Lancy:
\begin{center}
少年应有鸿鹄志,当骑骏马踏平川
\end{center}
\noindent $N_2$:
\begin{center}
I see,I like,I go,I get
\end{center}
\noindent Meursault:
\begin{center}
被拽进角斗场去面对一场殊死
搏斗和自己昂首走进去是不同的。
\end{center}
\tableofcontents

\part{模型与技巧}

\chapter{核心模型与定理}

\section{柯西不等式}
\subsection{代数形式}
对于实数序列 \(a_1,a_2,\dots,a_n\) 和 \(b_1,b_2,\dots,b_n\),有:
\[
\left(\sum_{i=1}^n a_i^2\right)\left(\sum_{i=1}^n b_i^2\right) \geq \left(\sum_{i=1}^n a_i b_i\right)^2
\]
当且仅当 \(\frac{a_1}{b_1} = \frac{a_2}{b_2} = \dots = \frac{a_n}{b_n}\)(分母不为零)时等号成立。

\subsection{二维特殊形式}
\[
(a_1^2 + a_2^2)(b_1^2 + b_2^2) \geq (a_1 b_1 + a_2 b_2)^2
\]

\subsection{几何解释}
在向量空间中,设向量 \(\vec{a} = (a_1,a_2,\dots,a_n)\),\(\vec{b} = (b_1,b_2,\dots,b_n)\),则:
\[
|\vec{a} \cdot \vec{b}| \leq \|\vec{a}\| \cdot \|\vec{b}\|
\]
其中 \(\|\vec{a}\| = \sqrt{\sum_{i=1}^n a_i^2}\) 为向量的模长。等号成立当且仅当 \(\vec{a}\) 与 \(\vec{b}\) 线性相关(平行)。

几何意义:两个向量的点积绝对值不超过模长乘积,反映了夹角 \(\theta\) 满足 \(|\cos\theta| \leq 1\)(因 \(\vec{a} \cdot \vec{b} = \|\vec{a}\| \|\vec{b}\| \cos\theta\))。

\subsection{例题}
已知实数 \(a,b,c\) 满足约束条件 \(3a^2 + 3b^2 + 4c^2 = 60\),求 \(a + b + c\) 的最大值。

解答:
由柯西不等式:
\[
(a + b + c)^2 = \left(\sqrt{3}a \cdot \frac{1}{\sqrt{3}} + \sqrt{3}b \cdot \frac{1}{\sqrt{3}} + 2c \cdot \frac{1}{2}\right)^2
\]
\[
\leq \left(3a^2 + 3b^2 + 4c^2\right)\left(\frac{1}{3} + \frac{1}{3} + \frac{1}{4}\right)
\]
代入 \(3a^2 + 3b^2 + 4c^2 = 60\):
\[
(a + b + c)^2 \leq 60 \times \frac{11}{12} = 55
\]
故最大值为 \(\sqrt{55}\)。

\section{阿波罗尼斯圆}
\subsection{定义与构造}
平面内到两个定点 \(A,B\) 的距离之比为常数 \(k(k \neq 1)\) 的点的轨迹是圆,称为阿波罗尼斯圆。

构造步骤:
\\ \indent 1. 内分点 \(C\):在线段 \(AB\) 上,满足 \(\frac{AC}{BC} = k\),则 \(AC = \frac{k}{1+k}AB\),\(BC = \frac{1}{1+k}AB\)。
\\ \indent 2. \(D\):在直线 \(AB\) 延长线上(\(B\) 外侧),满足 \(\frac{AD}{BD} = k\),则 \(BD = \frac{AB}{k-1}\),\(AD = \frac{k}{k-1}AB\)。
\\ \indent 3.迹圆:以 \(CD\) 为直径的圆(因内角平分线与外角平分线垂直,\(\angle CPD = 90^\circ\))。

\begin{figure}[htbp] 
    \centering 
    \includegraphics[width=0.8\textwidth]{1.png}  
\end{figure}

 
\subsection{坐标法证明}
(略,可通过距离公式直接推导)

\section{阿基米德三角形}
\subsection{定义}
过抛物线 \(x^2 = 2py\)(\(p > 0\))外一点 \(P(x_0,y_0)\) 作抛物线的两条切线 \(PA,PB\)(切点为 \(A,B\)),则 \(\triangle PAB\) 称为阿基米德三角形。

\subsection{核心性质}
1. \(k_{AB} \cdot k_{PQ} = \frac{2y_0}{p}\)(\(Q\) 为 \(AB\) 与 \(y\) 轴交点)\\
\indent 2. $ k_{PA}\cdot k_{PQ}=\frac{2y_0}{p}$\\ 
\indent 3. \(PM \parallel y\) 轴(\(M\) 为 \(AB\) 中点)\\
\indent 4. \(PM\) 的中点 \(R\) 在抛物线上,且 \(AB \parallel R\) 处的切线

\begin{figure}[H] 
    \centering 
    \includegraphics[width=0.6\textwidth]{2.png}  
\end{figure}

\subsection{简单证明}
设切点 \(A(x_1,\frac{x_1^2}{2p})\),\(B(x_2,\frac{x_2^2}{2p})\),抛物线切线方程为 \(xx_1 = p(y + \frac{x_1^2}{2p})\),即 \(x_1 x = py + \frac{x_1^2}{2}\)。

因 \(P(x_0,y_0)\) 在切线上,故 \(x_1 x_0 = p y_0 + \frac{x_1^2}{2}\),同理 \(x_2 x_0 = p y_0 + \frac{x_2^2}{2}\)。

故 \(x_1,x_2\) 是方程 \(x^2 - 2x_0 x + 2p y_0 = 0\) 的根,得 \(x_1 + x_2 = 2x_0\),\(x_1 x_2 = 2p y_0\)。

直线 \(AB\) 的方程:\(x(x_1 + x_2) = 2py + x_1 x_2\),代入根与系数关系可得性质(略)。

\section{不动点解决数列问题}
\subsection{不动点定义}
满足 \(x = f(x)\) 的根称为函数 \(y = f(x)\) 的不动点。若数列 \(\{a_n\}\) 满足 \(a_{n+1} = f(a_n)\),且 \(x_0\) 为不动点,则 \(a_n = x_0 \implies a_{n+1} = x_0\)(常数列)。

\subsection{核心性质}
若 \(x_0\) 为 \(f(x)\) 的不动点,且 \(f(x)\) 为多项式或分式线性函数,则:
\[
f(x) - x_0 = (x - x_0) \cdot A(x)
\]
其中 \(A(x)\) 为某一因式。

\subsection{解题思路}
对难以直接求解的递推数列,通过不动点构造等比数列或等差数列,常结合取倒数法。

\subsection{例题}
已知 \(a_{n+1} = \frac{3a_n + 1}{a_n + 3}\),\(a_1 = 2\),求通项。

解答:
求不动点:\(x = \frac{3x + 1}{x + 3} \implies x^2 = 1 \implies x = \pm 1\)。

构造 \(\frac{a_{n+1} - 1}{a_{n+1} + 1} = \frac{\frac{3a_n + 1}{a_n + 3} - 1}{\frac{3a_n + 1}{a_n + 3} + 1} = \frac{2(a_n - 1)}{4(a_n + 1)} = \frac{1}{2} \cdot \frac{a_n - 1}{a_n + 1}\)。

故 \(\left\{\frac{a_n - 1}{a_n + 1}\right\}\) 是等比数列,首项 \(\frac{2 - 1}{2 + 1} = \frac{1}{3}\),公比 \(\frac{1}{2}\),得:
\[
\frac{a_n - 1}{a_n + 1} = \frac{1}{3} \cdot \left(\frac{1}{2}\right)^{n-1} \implies a_n = \frac{3 \cdot 2^{n-1} + 1}{3 \cdot 2^{n-1} - 1}
\]

\section{三角形的角平分线与中线}
\subsection{角平分线相关}
1. 角平分线定理:\(\frac{AB}{AC} = \frac{BD}{DC}\)(\(AD\) 为 \(\angle BAC\) 平分线)。
\\ \indent 2. 面积关系:\(\frac{S_{\triangle ABD}}{S_{\triangle ACD}} = \frac{AB}{AC}\)。
\\ \indent 3. 长度公式:\(AD^2 = AB \cdot AC - BD \cdot CD\)。

\begin{figure}[htbp] 
    \centering 
    \includegraphics[width=0.4\textwidth]{3.png}  
\end{figure}

\subsection{中线相关}
设 \(BC = d\),中线 \(AD = c\),则:
\[
2(a^2 + b^2) = c^2 + d^2
\]
证明方法:
\\ \indent 1. 向量法:\(\overrightarrow{AD} = \frac{1}{2}(\overrightarrow{AB} + \overrightarrow{AC})\),平方后展开。
\\ \indent 2. 余弦定理:在 \(\triangle ABD\) 和 \(\triangle ADC\) 中分别应用余弦定理,相加消去夹角项。

\begin{figure}[htbp] 
    \centering 
    \includegraphics[width=0.4\textwidth]{4.png}  
\end{figure}

\section{用向量求三角形的面积}
\subsection{向量外积公式}
对于平面向量 \(\vec{a} = (x_1,y_1)\),\(\vec{b} = (x_2,y_2)\),以 \(\vec{a},\vec{b}\) 为邻边的平行四边形面积为:
\[
S = |x_1 y_2 - x_2 y_1|
\]
三角形面积为其一半:
\[
S_{\triangle} = \frac{1}{2}|x_1 y_2 - x_2 y_1|
\]

\subsection{坐标法应用}
若三角形顶点为 \(A(x_1,y_1)\),\(B(x_2,y_2)\),\(C(x_3,y_3)\),则:
\[
S_{\triangle ABC} = \frac{1}{2}|\overrightarrow{AB} \times \overrightarrow{AC}| = \frac{1}{2}|(x_2 - x_1)(y_3 - y_1) - (x_3 - x_1)(y_2 - y_1)|
\]

\section{三倍角定理与数列求和技巧}
\subsection{三倍角公式}
\[
\sin 3A = 3\sin A - 4\sin^3 A
\]
\[
\cos 3A = 4\cos^3 A - 3\cos A
\]

\subsection{错位相减法}
\subsubsection{适用范围}
求数列 \(\{a_n \cdot b_n\}\) 的前 \(n\) 项和,其中 \(\{a_n\}\) 为等差数列,\(\{b_n\}\) 为等比数列(公比 \(q \neq 1\))。

\subsubsection{基本步骤}
设 \(S_n = a_1 b_1 + a_2 b_2 + \dots + a_n b_n\),则:
1. 两边乘公比:\(q S_n = a_1 b_2 + a_2 b_3 + \dots + a_n b_{n+1}\)。
\\ \indent 2. 两式相减:\((1 - q)S_n = a_1 b_1 + d(b_2 + \dots + b_n) - a_n b_{n+1}\)(\(d\) 为等差数列公差)。
\\ \indent 3. 计算等比数列和:\(b_2 + \dots + b_n = b_1 q \cdot \frac{1 - q^{n-1}}{1 - q}\)。
\\ \indent 4. 化简得 \(S_n\)。

\subsubsection{例题}
已知 \(a_n = (2n - 1) \cdot 3^n\),求 \(S_n\)。

解答:
\[
S_n = 1 \cdot 3 + 3 \cdot 3^2 + 5 \cdot 3^3 + \dots + (2n - 1) \cdot 3^n
\]
\[
3S_n = 1 \cdot 3^2 + 3 \cdot 3^3 + \dots + (2n - 3) \cdot 3^n + (2n - 1) \cdot 3^{n+1}
\]
\indent 相减得:
\[
-2S_n = 3 + 2(3^2 + 3^3 + \dots + 3^n) - (2n - 1) \cdot 3^{n+1}
\]
\[
= 3 + 2 \cdot \frac{9(3^{n-1} - 1)}{2} - (2n - 1) \cdot 3^{n+1}
\]
\[
= 3 + 3^{n+1} - 9 - (2n - 1) \cdot 3^{n+1}
\]
\[
= -6 + (2 - 2n) \cdot 3^{n+1}
\]
故 \(S_n = 3 + (n - 1) \cdot 3^{n+1}\)。

\subsection{裂项相消法}
已知数列通项\(a_n=(2n-1)3^n\),求前n项和\(S_n\)
\subsubsection{核心思想}
构造 \(a_n = f(n+1) - f(n)\),利用累加法求和:
\[
S_n = f(n+1) - f(1)
\]

\subsubsection{例题}
同错位相减例题,用裂项法求解。

解答:
设 \(a_n = 3^{n+1}(A(n+1) + B) - 3^n(An + B)\),展开对比系数得 \(A = 1\),\(B = -2\),故:
\[
a_n = 3^{n+1}(n - 1) - 3^n(n - 2)
\]
\[
S_n = [3^2 \cdot 0 - 3^1 \cdot (-1)] + [3^3 \cdot 1 - 3^2 \cdot 0] + \dots + [3^{n+1}(n - 1) - 3^n(n - 2)]
\]
\[
= 3 + (n - 1) \cdot 3^{n+1}
\]

\section{立体几何法向量简单求法}
对于空间向量 \(\vec{a} = (x_1,y_1,z_1)\),\(\vec{b} = (x_2,y_2,z_2)\),其叉乘(外积)即为平面的法向量:
\[
\vec{a} \times \vec{b} = \begin{vmatrix}
\vec{i} & \vec{j} & \vec{k} \\
x_1 & y_1 & z_1 \\
x_2 & y_2 & z_2
\end{vmatrix} = (y_1 z_2 - y_2 z_1, z_1 x_2 - z_2 x_1, x_1 y_2 - x_2 y_1)
\]

快速计算技巧:
1. 将向量坐标写两遍:\(x_1,y_1,z_1,x_1,y_1,z_1\) 和 \(x_2,y_2,z_2,x_2,y_2,z_2\)。
2. 划去首尾列,按二阶行列式计算:
\[
\text{法向量} = (y_1 z_2 - y_2 z_1, z_1 x_2 - z_2 x_1, x_1 y_2 - x_2 y_1)
\]

\section{拉乘加法(因式分解技巧)}

\begin{figure}[H] 
    \centering 
    \includegraphics[width=0.8\textwidth]{5.png}  
\end{figure}

用于多项式快速分解,步骤:
\\ \indent\hspace{2em} 1. 猜测有理根 \(x = k\)(利用有理根定理:可能根为常数项因数与首项系数因数之比)。
\\ \indent\hspace{2em} 2. 若 \(f(k) = 0\),则 \((x - k)\) 为因式。
\\ \indent\hspace{2em} 3. 用多项式除法或配方法分解剩余因式。

例题:分解 \(f(x) = x^3 - 6x^2 + 11x - 6\)。

解答:
猜测根 \(x = 1\),\(f(1) = 1 - 6 + 11 - 6 = 0\),故 \((x - 1)\) 为因式。
分解得:\(f(x) = (x - 1)(x^2 - 5x + 6) = (x - 1)(x - 2)(x - 3)\)。

\section{一动点到两定点距离平方和}
核心公式:设 \(M\) 为 \(AB\) 中点,\(P\) 为平面内任意一点,则:
\[
PA^2 + PB^2 = 2PM^2 + 2MA^2
\]

例题:已知 \(|\vec{a}| = 1\),\(|\vec{b}| = 2\),\(2\vec{a}^2 = \vec{b} \cdot \vec{c}\),\(2\vec{b}^2 = \vec{a} \cdot \vec{c}\),求 \(|\vec{a} - \vec{b} + \vec{c}|^2\) 的最小值。

解答:
由 \(2\vec{a}^2 = \vec{b} \cdot \vec{c}\) 得 \(\vec{c} \cdot (\vec{b} - 2\vec{a}) = 0\),故 \(\vec{c}\) 的终点在以 \(\vec{b}\) 为直径的圆上。
\[
|\vec{a} - \vec{b} + \vec{c}|^2 = |(\vec{c} - \vec{a}) + (\vec{a} - \vec{b})|^2 = 2|\vec{c} - M|^2 + 2|MA|^2
\]
(\(M\) 为 \(\vec{a}\) 与 \(\vec{b}\) 中点),计算得最小值为 \(7 - 2\sqrt{3}\)。

\section{求异面直线的距离}

\begin{figure}[H] 
    \centering 
    \includegraphics[width=0.4\textwidth]{6.png}  
\end{figure}

核心公式:设异面直线 \(l_1,l_2\) 的公垂线段方向向量为 \(\vec{n}\),\(C \in l_1\),\(D \in l_2\),则距离:
\[
d = \frac{|\overrightarrow{CD} \cdot \vec{n}|}{|\vec{n}|}
\]

几何构造:作 \(l_1' \parallel l_1\) 且 \(l_1' \cap l_2 = O\),则距离转化为 \(l_1\) 上一点到平面 \(l_1' \cup l_2\) 的距离。

\section{三余弦定理与三正弦定理}
\subsection{三余弦定理(最小角定理)}
在平面内,若 \(BD \perp AB\),\(CD \perp BD\),则:
\[
\cos \angle ABC = \cos \angle ABD \cdot \cos \angle DBC
\]

\begin{figure}[H] 
    \centering 
    \includegraphics[width=0.6\textwidth]{7.png}  
\end{figure}

\subsection{三正弦定理(二面角最大)}
在空间中,设平面 \(\alpha \perp \beta\),交线为 \(l\),\(A \in \alpha\),\(B \in l\),\(C \in \beta\),则:
\[
\sin \alpha = \sin \beta \cdot \sin \gamma
\]
(\(\alpha\) 为 \(AC\) 与 \(\beta\) 的夹角,\(\beta\) 为 \(AC\) 与 \(l\) 的夹角,\(\gamma\) 为二面角)。

\begin{figure}[H] 
    \centering 
    \includegraphics[width=0.6\textwidth]{8.png}  
\end{figure}

\section{泰勒展开与麦克劳林公式}
\subsection{泰勒公式}
若 \(f(x)\) 在 \(x = x_0\) 处有 \(n\) 阶导数,在区间 \([a,b]\) 上有 \(n+1\) 阶导数,则:
\[
f(x) = f(x_0) + f'(x_0)(x - x_0) + \frac{f''(x_0)}{2!}(x - x_0)^2 + \dots + \frac{f^{(n)}(x_0)}{n!}(x - x_0)^n + R_n(x)
\]
其中 \(R_n(x)\) 为余项。

\subsection[麦克劳林展开]{(\(x_0 = 0\))}
1. \(e^x = 1 + x + \frac{x^2}{2!} + \frac{x^3}{3!} + R_n(x) \)
\\ \indent 2. \(\ln(x + 1) = x - \frac{x^2}{2} + \frac{x^3}{3} - \dots + (-1)^{n-1}\frac{x^n}{n} + \dots\)(\(x > -1\))
\\ \indent 3. \(\sin x = x - \frac{x^3}{3!} + \frac{x^5}{5!} - \dots + (-1)^n\frac{x^{2n+1}}{(2n+1)!} + \dots\)
\\ \indent 4. \(\cos x = 1 - \frac{x^2}{2!} + \frac{x^4}{4!} - \dots + (-1)^n\frac{x^{2n}}{(2n)!} + \dots\)

\subsection{常用不等式推导}
1. \(e^x \geq 1 + x\)(\(x \in \mathbb{R}\))
\\ \indent 2. \(e^x \geq 1 + x + \frac{x^2}{2}\)(\(x \geq 0\))
\\ \indent 3. \(x - \frac{x^2}{2} \leq \ln(x + 1) \leq x\)(\(x > 0\))
\\ \indent 4. \(e^x + e^{-x} \geq x^2 + 2\)(\(x \in \mathbb{R}\))

\section{重要不等式}
\subsection{基本不等式}
\(a + b \geq 2\sqrt{ab}\)(\(a,b > 0\),当且仅当 \(a = b\) 等号成立)。

\subsection{柯西不等式}
(见本章第一节)

\subsection{伯努利不等式}
\((1 + x)^n \geq 1 + nx\)(\(x > -1\),\(n \in \mathbb{N}^*\))。

\subsection{对数不等式}
1. \(x > 1\) 时,\(\ln x > \frac{2(x - 1)}{x + 1}\)
\\ \indent 2. \(x > 1\) 时,\(\ln x \leq \sqrt{x} - \frac{1}{\sqrt{x}}\)
\\ \indent 3. \(x > 0\) 时,\(\ln x \leq x - 1\)

\section{极点极线,调和点列,定比点差}
\subsection{调和点列定义}
若点列 \(A,B,M,N\) 满足 \(\frac{AM}{MB} = \frac{AN}{NB} = \lambda\),则称其为调和点列。

\subsection{核心性质}
1. 若 \(A,M,B,N\) 为调和点列,则 \(M\) 的极线过 \(N\)。
\\ \indent 2. 若 \(M\) 的极线过 \(N\),则 \(A,B,M,N\) 为调和点列。

\subsection{定比点差法}
设椭圆 \(\frac{x^2}{a^2} + \frac{y^2}{b^2} = 1\),点 \(A(x_1,y_1),B(x_2,y_2)\) 在椭圆上,\(M(x_M,y_M)\) 满足 \(\overrightarrow{AM} = \lambda \overrightarrow{MB}\),则:
\[
x_M = \frac{x_1 + \lambda x_2}{1 + \lambda}, \quad y_M = \frac{y_1 + \lambda y_2}{1 + \lambda}
\]
将 \(A,B\) 代入椭圆方程,两式相减并代入 \(M\) 坐标,可得极线方程 \(\frac{x_M x}{a^2} + \frac{y_M y}{b^2} = 1\)。

\section{奔驰定理与四心}
\subsection{奔驰定理}

\begin{figure}[htbp] 
    \centering 
    \includegraphics[width=0.6\textwidth]{9.png}  
\end{figure}

在 \(\triangle ABC\) 中,设 \(S_A = S_{\triangle PBC}\),\(S_B = S_{\triangle PAC}\),\(S_C = S_{\triangle PAB}\),则:
\[
S_A \cdot \overrightarrow{PA} + S_B \cdot \overrightarrow{PB} + S_C \cdot \overrightarrow{PC} = \vec{0}
\]

\subsection{四心的比例关系}
1. 重心:\(S_A : S_B : S_C = 1:1:1\)
\\ \indent 2. 垂心:\(S_A : S_B : S_C = \tan A : \tan B : \tan C\)
\\ \indent 3. 外心:\(S_A : S_B : S_C = \sin 2A : \sin 2B : \sin 2C\)
\\ \indent 4. 内心:\(S_A : S_B : S_C = a : b : c\)(\(a,b,c\) 为三角形三边)

\section{过焦点直线与圆锥曲线}

\begin{figure}[htbp] 
    \centering 
    \includegraphics[width=0.9\textwidth]{10.png}  
\end{figure}

\subsection[椭圆] {\(\frac{x^2}{a^2} + \frac{y^2}{b^2} = 1\)(\(a > b > 0\))}
1. 焦半径:\(|PF_1| = a + ex_0\),\(|PF_2| = a - ex_0\)(\(F_1,F_2\) 为焦点,\(P(x_0,y_0)\) 在椭圆上)
\\ \indent 2. 焦点弦长:\(|AB| = \frac{2ab^2}{a^2 - c^2 \cos^2 \alpha}\)(\(\alpha\) 为直线倾斜角)
\\ \indent 3. 斜率乘积:\(k_{PA} \cdot k_{PB} = -\frac{b^2}{a^2}\)(特殊情况)

\subsection[双曲线] {\(\frac{x^2}{a^2} - \frac{y^2}{b^2} = 1\)(\(a > 0,b > 0\))}
1. 焦半径(右支):\(|PF_1| = ex_0 + a\),\(|PF_2| = ex_0 - a\)
\\ \indent 2. 焦点弦长:\(|AB| = \frac{2ab^2}{|a^2 \cos^2 \alpha - c^2|}\)

\subsection[抛物线] {\(y^2 = 2px\)(\(p > 0\))}
1. 焦半径:\(|PF| = x_0 + \frac{p}{2}\)
\\ \indent 2. 焦点弦性质:
   \\ \indent\hspace{2em}- \(y_1 y_2 = -p^2\),\(x_1 x_2 = \frac{p^2}{4}\)
   \\ \indent\hspace{2em}- 以 \(AB\) 为直径的圆与准线 \(x = -\frac{p}{2}\) 相切
   \\ \indent\hspace{2em}- 以 \(FP\) 为直径的圆与 \(y\) 轴相切

\section{三角函数结论}
图像缩放性质
对于 \(y = A\sin(\omega x + \varphi)\):
- 振幅 \(A\) 决定最大值:\(y_{\text{max}} = |A|\)
- 周期 \(T = \frac{2\pi}{|\omega|}\)
- 缩放关系:\(\frac{A}{y_{\text{max}}} = 1\)(标准正弦曲线 \(A = 1\),\(y_{\text{max}} = 1\))

\begin{figure}[H] 
    \centering 
    \includegraphics[width=0.8\textwidth]{12.png}  
\end{figure}

\section{拐点与函数凹凸性}
\subsection{凹凸性定义}
设 \(x_1,x_2 \in I\),\(x_1 \neq x_2\):
1. 若 \(f\left(\frac{x_1 + x_2}{2}\right) < \frac{f(x_1) + f(x_2)}{2}\),则 \(f(x)\) 在 \(I\) 上为下凸(凹)函数。
\\ \indent 2. 若 \(f\left(\frac{x_1 + x_2}{2}\right) > \frac{f(x_1) + f(x_2)}{2}\),则 \(f(x)\) 在 \(I\) 上为上凸(凸)函数。

\subsection{凹凸性与二阶导数}
- 下凸函数:\(f''(x) > 0\)
\\ \indent - 上凸函数:\(f''(x) < 0\)

\subsection{拐点定义}
若 \(f''(x_0) = 0\) 且 \(f''(x)\) 在 \(x_0\) 两侧变号,则 \((x_0,f(x_0))\) 为拐点;二阶导数不存在的点也可能是拐点。

\section{米勒定理}
定理内容
在平面上,点 \(A,B\) 为定点,\(P\) 为动点,当 \(\angle APB\) 最大时,\(P\) 在过 \(A,B\) 且与动点轨迹相切的圆上。

几何意义
最大角对应圆的切线位置,易证(略)。

\begin{figure}[H] 
    \centering 
    \includegraphics[width=0.7\textwidth]{13.png}  
\end{figure}

\section{极坐标}
基本定义
平面内点 \(P\) 的极坐标为 \((\rho,\theta)\),其中 \(\rho\) 为极径(到极点距离),\(\theta\) 为极角(与极轴夹角)。

与直角坐标转换
\[
x = \rho \cos \theta, \quad y = \rho \sin \theta
\]
\[
\rho = \sqrt{x^2 + y^2}, \quad \theta = \arctan\frac{y}{x}
\]

圆锥曲线极坐标方程
\\ \indent\hspace{2em} 1. 椭圆:\(\rho = \frac{ep}{1 - e\cos\theta}\)(\(e < 1\))
\\ \indent\hspace{2em} 2. 双曲线:\(\rho = \frac{ep}{1 - e\cos\theta}\)(\(e > 1\))
\\ \indent\hspace{2em} 3. 抛物线:\(\rho = \frac{p}{1 - \cos\theta}\)(\(e = 1\))

\begin{figure}[H] 
    \centering 
    \includegraphics[width=0.7\textwidth]{14.png}  
\end{figure}

\section{组合数的推论}
基本公式
\\ \indent\hspace{2em} 1. \(C_n^k = C_n^{n - k}\)
\\ \indent\hspace{2em} 2. \(C_n^k + C_n^{k - 1} = C_{n + 1}^k\)
\\ \indent\hspace{2em} 3. \(kC_n^k = nC_{n - 1}^{k - 1}\)

常用推论
\[
\sum_{k=0}^n C_n^k = 2^n, \quad \sum_{k=0}^n (-1)^k C_n^k = 0
\]

\begin{figure}[H] 
    \centering 
    \includegraphics[width=0.7\textwidth]{15.png}  
\end{figure}

\section{圆锥曲线方程联立快速结论}
设椭圆 \(\frac{x^2}{a^2} + \frac{y^2}{b^2} = 1\) 与直线 \(Ax + By + C = 0\) 联立,消去 \(y\) 得 \(mx^2 + nx + p = 0\),则:
\\ \indent\hspace{2em}1. 根与系数:\(x_1 + x_2 = -\frac{n}{m}\),\(x_1 x_2 = \frac{p}{m}\)
\\ \indent\hspace{2em}2. 弦中点:\((x_0,y_0) = \left(-\frac{n}{2m}, -\frac{A n}{2B m} - \frac{C}{B}\right)\)
\\ \indent\hspace{2em}3. 弦长:\(|PQ| = \frac{2ab\sqrt{(1 + m^2)(a^2 m^2 + b^2 - n^2)}}{a^2 m^2 + b^2}\)(直线斜率为 \(m\) 时)
\\ \indent\hspace{2em}4. 斜率乘积:\(k_{OP} \cdot m = -\frac{b^2}{a^2}\)(定值,\(O\) 为原点)

\part{学长的考前必刷题(增强题感)}

\chapter{必刷20题}

\section{题目1}
在 \(\triangle ABC\) 中,内角 \(A,B,C\) 所对的边分别为 \(a,b,c\),\(G\) 为 \(\triangle ABC\) 的重心。
\begin{enumerate}
  \item 若 \(AG \perp BG\),\(\frac{1}{tan A}+\frac{1}{tan B} = \frac{2\lambda}{tan C}\),求 \(\lambda\) 的值;
  \item 若 \(B = 60^\circ\),\(2b = a + c\),判断 \(\triangle ABC\) 的形状;
  \item 在(2)的条件下,\(a = 2\),\(M、N\) 是边 \(AB、AC\) 上的两点(含端点),且满足 \(\overrightarrow{AG} = t\overrightarrow{AM} + (1 - t)\overrightarrow{AN}\),求 \(\frac{1}{GM^2} + \frac{1}{GN^2}\) 的取值范围。
\end{enumerate}

\subsection{答案}
\begin{enumerate}
\item \(\lambda = 1\);
\item \(\triangle ABC\) 是等边三角形;
\item 取值范围是 \(\left[\frac{3}{2}, 3\right]\)。
\end{enumerate}

\section{题目2}
在 \(\triangle ABC\) 中,角 \(A,B,C\) 所对的边分别是 \(a,b,c\),且满足 \(a \cos C + \sqrt{3}a \sin C - b - c = 0\)。
\begin{enumerate}
\item 求角 \(A\);
\item 若 \(a = \sqrt{3}\),求 \(\triangle ABC\) 周长的最大值;
\item 求 \(\frac{bc-ab-ac}{a^2}\) 的取值范围。
\end{enumerate}

\subsection{答案}
\begin{enumerate}
\item \(A = \frac{\pi}{3}\);
\item 周长最大值为 \(3\sqrt{3}\);
\item 取值范围是 \((-\infty, -\frac{3}{4}]\)。
\end{enumerate}

\section{题目3}
已知数列 \(\{a_n\}\) 和 \(\{b_n\}\),\(a_1 = 2\),\(b_n = 1 - \frac{1}{a_n}\)(\(n \in \mathbb{N}^*\)),函数 \(f(x) = \ln(1 + x) - \frac{mx}{x + 1}\),其中 \(m > 0\)。
\begin{enumerate}
\item 求函数 \(f(x)\) 的单调区间;
\item 若数列 \(\{a_n\}\) 各项均为正整数,且对任意的 \(n \in \mathbb{N}^*\) 都 \(|a_{n+1} - \frac{2a_n^2+a_{n+1}}{a_n+1} | < \frac{1}{2} \),求证:
\begin{enumerate}
\item \(a_{n+1} = 2a_n\)(\(n \in \mathbb{N}^*\));
\item \(b_1 b_2 \dots b_n > e^{-\frac{3}{2}}\),其中 \(e = 2.71828\cdots\) 为自然对数的底数。
\end{enumerate}
\end{enumerate}

\subsection{答案}
\begin{enumerate}
\item 单调递减区间为 \((-1, m - 1)\),单调递增区间为 \((m - 1, +\infty)\);
\item 证明略。
\end{enumerate}

\section{题目4}
设数列 \(\{a_n\}\) 的前 \(n\) 项之积为 \(T_n\),满足 \(2a_n + T_n = 1\)(\(n \in \mathbb{N}^*\))。
\begin{enumerate}
\item 设 \(b_n = 1 + \frac{1}{T_n}\),求数列 \(\{b_n\}\) 的通项公式;
\item 设数列 \(\{a_n\}\) 的前 \(n\) 项之和为 \(S_n\),证明:\(\frac{n}{2} + \frac{1}{2}\ln(T_n + 1) - \frac{1}{3} < S_n < \frac{n}{2} + \frac{1}{2}\ln(T_n + 1) - \frac{1}{4}\)。
\end{enumerate}

\subsection{答案}
\begin{enumerate}
\item \(b_n = 2^{n + 1}\);
\item 证明略。
\end{enumerate}

\section{题目5}
(题目图略)
设函数 \(f(x) = \ln x - \frac{2(x - 1)}{x + 1}\)。
\begin{enumerate}
\item 证明:函数f(x)在定义域内存在唯一零点;
\item 设 \(b > a > 0\),求证:\(\frac{b+a}{2} > \frac{b - a}{ln b - ln a}\);
\item 设 数列\({a_n}\)的通项\(a_n=1+\frac{1}{2}+\frac{1}{3}+ \cdots + \frac{1}{n} \),,求证:\(\ln(2n + 1) > a_n\)。
\end{enumerate}

\subsection{答案}
\begin{enumerate}
\item 证明略;
\item 证明略;
\item 证明略。
\end{enumerate}

\section{题目6}
已知函数 \(f(x) = x e^{ax} - e^{x}\)。
\begin{enumerate}
\item 当 \(a = 1\) 时,讨论 \(f(x)\) 的单调性;
\item 当 \(x > 0\) 时,\(f(x) < -1\),求 \(a\) 的取值范围;
\item 设 \(n \in \mathbb{N}^*\),证明:\(\sqrt{\frac{1}{n^2 + 1}} + \sqrt{\frac{1}{n^2 + 2}} + \dots + \sqrt{\frac{1}{n^2 + n}} > \ln(n + 1)\)。
\end{enumerate}

\subsection{答案}
\begin{enumerate}
\item 单调递增区间为 \((-\infty, 0)\),单调递减区间为 \((0, +\infty)\);
\item \(a \geq -1\);
\item 证明略。
\end{enumerate}

\section{题目7}
设函数 \(f(x) = \frac{1+ln(x+1)}{x} \)(\(x > 0\))。
\begin{enumerate}
\item 若 \(f(x) > \frac{k}{x+1}\) 恒成立,求整数 \(k\) 的最大值;
\item 求证:\((1 + 1 \times 2) \cdot (1 + 2 \times 3) \cdot \dots \cdot [1 + n \times (n + 1)] > e^{2n - 3}\)。
\end{enumerate}

\subsection{答案}
\begin{enumerate}
\item \(k_{\text{max}} = 3\);
\item 证明略。
\end{enumerate}

\section{题目8}
已知函数 \(f(x) = e^{x-2} - a \ln(x - 1)\)。
\begin{enumerate}
\item 当 \(a = 1\),求 \(f(x)\) 的最小值;
\item 令 \(g(x) = \frac{x} {f(x + 2) + aln(x + 1)} \),若存在 \(x_1 < x_2\),使得 \(g(x_1) = g(x_2)\),求证:\(\ln x_2 - \ln(1 - x_1) > \ln 3\)。
\end{enumerate}

\subsection{答案}
\begin{enumerate}
\item 最小值为 \(1\);
\item 证明略。
\end{enumerate}

\section{题目9}
已知函数 \(f(x) = \ln x - ax + \frac{b}{x}\),且正数 \(a,b\) 满足 \(\frac{a\sqrt{b}+b\sqrt{a}}{\sqrt{a}+\sqrt{b}} \geq \sqrt{2}\)。
\begin{enumerate}
\item 讨论 \(f(x)\) 的单调性;
\item 若 \(F(x) = \ln(x + m) - nx + e^x\) 的零点为 \(x_1,x_2\),且 \(m,n\) 满足 \(n > \frac{3}{2}\),\(n(1 - m) < e\),求证:\(x_1 + x_2 < \frac{2(m-mn+e)}{2n-1} \)(其中 \(e = 2.71828\cdots\) 是自然对数的底数)。
\end{enumerate}

\subsection{答案}
\begin{enumerate}
\item 当 \(a \in (0,\frac{1}{2}) \) 时,单调递增区间为 \( (\frac{-1+\sqrt{1-4a^2}}{-2a},\frac{-1-\sqrt{1-4a^2}}{-2a} )\),单调递减区间为 \((0,\frac{-1+\sqrt{1-4a^2}}{-2a}) \),\((\frac{-1-\sqrt{1-4a^2}}{-2a}),+\infty \);当 \(a \in[\frac{1}{2},+\infty) \) 时,单调递减区间为 \((0, +\infty)\)
\item 证明略。
\end{enumerate}

\section{题目10}
已知函数 \(f(x) = (x - 1)\ln(x + 2)\)。
\begin{enumerate}
\item 曲线 \(y = f(x)\) 在点 \(P(-1, 0)\) 处的切线为 \(l\),求证:曲线 \(y = f(x)\) 上的点都不在直线 \(l\) 的下方;
\item 若关于 \(x\) 的方程 \(f(x) = a\)(\(a\) 为实数)有不等实根 \(x_1,x_2\),求证:\(|x_1 - x_2| \leq 2 + a (\frac{2+ln3}{2ln3}) \)。
\end{enumerate}

\subsection{答案}
\begin{enumerate}
\item 证明略;
\item 证明略。
\end{enumerate}

\section{题目11}
已知函数 \(f(x) = \ln x - mx\)(\(m\) 为常数)。
\begin{enumerate}
\item 当 \(m = 1\) 时,求曲线 \(y = f(x)\) 在点 \((1, f(1))\) 处的切线方程;
\item 当 \(m \geq \frac{3 \sqrt{2}}{2}\) 时,设函数 \(g(x) = 2f(x) + x^2\) 的两个极值点 \(x_1,x_2\)(\(x_1 < x_2\))恰满足关系式 \(b = \frac{lnx_1 - lnx_2}{x_1-x_2}\),求 \(y = (x_1 - x_2)(\frac{2}{x_1+x_2}-b)\) 的最小值。
\end{enumerate}

\subsection{答案}
\begin{enumerate}
\item 切线方程为 \(y = -1\);
\item 最小值为 \(-\frac{2}{3}+ln2\)。
\end{enumerate}

\section{题目12}
(同题目6,略)

\section{题目13}
已知曲线 \(C: \frac{x^2}{a^2} + \frac{y^2}{b^2} = t\)(\(a > b > 0\)),当 \(t(t > 0)\) 变化时得到一系列的椭圆,称为“\(a - b\) 椭圆群”。
\begin{enumerate}
\item 求“2 - 1 椭圆群”中椭圆的离心率;
\item 若“\(a - b\) 椭圆群”中的两个椭圆 \(C_1、C_2\) 对应的 \(t\) 分别为 \(t_1、t_2\),且 \(t_1 = 2t_2\)(\(t_2 > 0\)),则称 \(C_1、C_2\) 为“和谐椭圆对”。已知 \(C_1、C_2\) 为“和谐椭圆对”,\(P\) 是 \(C_2\) 上的任意一点,过点 \(P\) 作 \(C_2\) 的切线交 \(C_1\) 于 \(A、B\) 两点,\(Q\) 为 \(C_1\) 上异于 \(A、B\) 的任意一点,且满足 \(\overrightarrow{OQ} = \alpha \overrightarrow{OA} + \beta \overrightarrow{OB}\),问:\(\alpha^2 + \beta^2\) 是否为定值?若为定值,求出该定值;否则,说明理由。
\end{enumerate}

\subsection{答案}
\begin{enumerate}
\item 离心率为 \(\frac{\sqrt{3}}{2}\);
\item 定值为 \(1\)。
\end{enumerate}

\section{题目14}
已知双曲线 \(C: \frac{x^2}{a^2} - \frac{y^2}{b^2} = 1\)(\(a,b > 0\))的左、右焦点分别为 \(F_1,F_2\),左顶点为 \(A(-2,0)\),点 \(M\) 为双曲线上一动点,且 \(|MF_1|^2 + |MF_2|^2\) 的最小值为 18,\(O\) 为坐标原点。
\begin{enumerate}
\item 求双曲线 \(C\) 的标准方程;
\item 已知直线 \(l: x = m\) 与 \(x\) 轴的正半轴交于点 \(T\),过点 \(T\) 的直线交双曲线 \(C\) 右支于点 \(B,D\),直线 \(AB,AD\) 分别交直线 \(l\) 于点 \(P,Q\),若 \(O,A,P,Q\) 四点共圆,求实数 \(m\) 的值。
\end{enumerate}

\begin{figure}[H] 
    \centering 
    \includegraphics[width=0.8\textwidth]{18.png}  
\end{figure}

\subsection{答案}
\begin{enumerate}
\item 标准方程为 \(\frac{x^2}{4} - y^2 = 1\);
\item \(m = \frac{2}{5}\)。
\end{enumerate}

\section{题目15}
如图,在平面直角坐标系 \(xOy\) 中,椭圆 \(C: \frac{x^2}{a^2} + \frac{y^2}{b^2} = 1\)(\(a > b > 0\))过点 \(A(-2,0)\),且离心率为 \(\frac{1}{2}\)。设椭圆 \(C\) 的右顶点为 \(B\),点 \(P,Q\) 是椭圆 \(C\) 上异于 \(A,B\) 的两个动点,记直线 \(AP,BQ\) 的斜率分别为 \(k_1,k_2\),且 \(3k_1 + k_2 = 0\)。
\begin{enumerate}
\item 求证:直线 \(PQ\) 过定点 \(R\);
\item 设直线 \(AQ,BP\) 相交于点 \(T\),记 \(\triangle ABT, \triangle ABQ\) 的面积分别为 \(S_1,S_2\),求 \(\frac{S_1}{S_2}\) 的取值范围。
\end{enumerate}

\begin{figure}[H] 
    \centering 
    \includegraphics[width=0.8\textwidth]{19.png}  
\end{figure}

\subsection{答案}
\begin{enumerate}
\item 证明略;
\item 取值范围是 \((\frac{3}{4}, 1)\)。
\end{enumerate}

\section{题目16}
已知椭圆 \(C: \frac{x^2}{a^2} + \frac{y^2}{b^2} = 1\)(\(a > b > 0\))的左顶点是 \(A\),右焦点是 \(F(1,0)\),过点 \(F\) 且斜率不为 0 的直线与 \(C\) 交于 \(M,N\) 两点,\(B\) 为线段 \(AM\) 的中点,\(O\) 为坐标原点,直线 \(AM\) 与 \(BO\) 的斜率之积为 \(-\frac{3}{4}\)。
\begin{enumerate}
\item 求椭圆 \(C\) 的方程;
\item 若直线 \(AM\) 和 \(AN\) 分别与直线 \(x = 4\) 交于 \(P,Q\) 两点,证明:以线段 \(PQ\) 为直径的圆恒过两个定点,并求出定点坐标。
\end{enumerate}

\subsection{答案}
\begin{enumerate}
\item 椭圆方程为 \(\frac{x^2}{4} + \frac{y^2}{3} = 1\);
\item 定点坐标为 \((1,0)\) 和 \((7,0)\)。
\end{enumerate}

\section{题目17}
如图,点 \(P\) 在抛物线 \(C: y = x^2\) 外,过 \(P\) 作抛物线 \(C\) 的两切线,设两切点分别为 \(A(x_1, x_1^2)\),\(B(x_2, x_2^2)\),记线段 \(AB\) 的中点为 \(M\)。
\begin{enumerate}
\item 求切线 \(PA,PB\) 的方程;
\item 设点 \(P\) 为圆 \(D: x^2 + (y + 2)^2 = 1\) 上的点,当 \(\frac{|AB|}{|PM|}\) 取最大值时,求点 \(P\) 的纵坐标。
\end{enumerate}

\begin{figure}[H] 
    \centering 
    \includegraphics[width=0.8\textwidth]{20.png}  
\end{figure}

\subsection{答案}
\begin{enumerate}
\item 切线方程:\(PA: y = 2x_1 x - x_1^2\),\(PB: y = 2x_2 x - x_2^2\);
\item 纵坐标为 \(-\frac{\sqrt{29}+1}{4}\)。
\end{enumerate}

\section{题目18}
已知椭圆 \(C: \frac{x^2}{a^2} + \frac{y^2}{b^2} = 1\)(\(a > b > 0\))的左、右焦点分别为 \(F_1,F_2\),离心率为 \(\frac{1}{2}\),\(P\) 是椭圆 \(C\) 上一动点,\(\triangle PF_1 F_2\) 面积的最大值为 \(\sqrt{3}\)。
\begin{enumerate}
\item 求椭圆 \(C\) 的标准方程;
\item 不过原点 \(O\) 的动直线 \(l\) 与椭圆 \(C\) 交于 \(A,B\) 两点,平面上一点 \(D\) 满足 \(\overrightarrow{OA} = \overrightarrow{AD}\),连接 \(BD\) 交椭圆 \(C\) 于点 \(E\)(点 \(E\) 在线段 \(BD\) 上且不与端点重合),若 \(\frac{\triangle EAB}{\triangle OAB}=\frac{2}{5} \),求原点 \(O\) 到直线 \(l\) 的距离的取值范围。
\end{enumerate}

\subsection{答案}
\begin{enumerate}
\item 标准方程为 \(\frac{x^2}{4} + \frac{y^2}{3} = 1\);
\item 取值范围是 \((\frac{\sqrt{6}}{2},\sqrt{2})\)。
\end{enumerate}

\section{题目19}
如图,\(A,B,C,D\) 是抛物线 \(E: y^2 = 4x\) 上的四个点(\(A,B\) 在 \(x\) 轴上方,\(C,D\) 在 \(x\) 轴下方),已知直线 \(AC\) 与 \(BD\) 的斜率分别为 \(-\frac{\sqrt{3}}{6}\) 和 \(2\),且直线 \(AC\) 与 \(BD\) 相交于点 \(P\)。
\begin{enumerate}
\item 若点 \(A\) 的横坐标为 6,则当 \(\triangle ADC\) 的面积取得最大值时,求点 \(D\) 的坐标;
\item 试判断 \(\frac{|PA|}{|PB|} \cdot \frac{|PC|}{|PD|}\) 是否为定值?若是,求出该定值;若不是,请说明理由。
\end{enumerate}

\begin{figure}[H] 
    \centering 
    \includegraphics[width=0.8\textwidth]{21.png}  
\end{figure}

\subsection{答案}
\begin{enumerate}
\item 点 \(D\) 的坐标为 \((\frac{3}{2}, -\sqrt{6})\);
\item 定值为 \(2\)。
\end{enumerate}

\section{题目20}
已知椭圆 \(C: \frac{x^2}{a^2} + \frac{y^2}{b^2} = 1\) 的焦距为 2,\(F_1,F_2\) 分别为左右焦点,过 \(F_1\) 的直线 \(l\) 与椭圆 \(C\) 交于 \(M,N\) 两点,\(\triangle F_2 MN\) 的周长为 8。
\begin{enumerate}
\item 求椭圆 \(C\) 的标准方程;
\item 已知结论:若点 \((x_0,y_0)\) 为椭圆 \(\frac{x^2}{a^2} + \frac{y^2}{b^2} = 1\) 上一点,则椭圆在该点的切线方程为 \(\frac{x_0 x}{a^2} + \frac{y_0 y}{b^2} = 1\)。点 \(T\) 为直线 \(x = 8\) 上的动点,过点 \(T\) 作椭圆 \(C\) 的两条不同切线,切点分别为 \(A,B\),直线 \(AB\) 交 \(x\) 轴于点 \(Q\),记 \(\triangle AF_1 Q\),\(\triangle BF_2 Q\) 的面积分别为 \(S_1,S_2\)。
\begin{enumerate}
\item 证明:\(Q\) 为定点;
\item 设 \(\mu = \frac{S_1}{S_2}\),求 \(\mu\) 的取值范围。
\end{enumerate}
\end{enumerate}

\subsection{答案}
\begin{enumerate}
\item 证明略;
\item ① 定点 \(Q(\frac{1}{2}, 0)\);② 取值范围是 \((\frac{9}{5}, 5)\)。
\end{enumerate}

\section{详细答案}
题目选自一个公众号“数栋的私人花园”。题目较好,公众号里面言论可能过激。请勿偏信!!!
\begin{figure}[H] 
    \centering 
    \includegraphics[width=0.8\textwidth]{22.png}  
    \includegraphics[width=0.8\textwidth]{23.png}  
\end{figure}
\begin{figure}[H] 
    \centering 
    \includegraphics[width=0.8\textwidth]{24.png}  
    \includegraphics[width=0.8\textwidth]{25.png}  
    \includegraphics[width=0.8\textwidth]{26.png}  
\end{figure}
\begin{figure}[H] 
    \centering 
    \includegraphics[width=0.8\textwidth]{27.png}  
    \includegraphics[width=0.8\textwidth]{28.png}   
    \includegraphics[width=0.8\textwidth]{29.png}  
\end{figure}
\begin{figure}[H] 
    \centering 
    \includegraphics[width=0.7\textwidth]{30.png}  
    \includegraphics[width=0.7\textwidth]{31.png}  
    \includegraphics[width=0.7\textwidth]{32.png}  
\end{figure}
\begin{figure}[H] 
    \centering 
    \includegraphics[width=0.8\textwidth]{33.png}  
    \includegraphics[width=0.8\textwidth]{34.png}  
    \includegraphics[width=0.8\textwidth]{35.png}  
    \includegraphics[width=0.8\textwidth]{36.png}  
\end{figure}
\begin{figure}[H] 
    \centering 
    \includegraphics[width=0.7\textwidth]{37.png}   
    \includegraphics[width=0.7\textwidth]{38.png}  
    \includegraphics[width=0.7\textwidth]{39.png}  
\end{figure}
原答案移步公众号“数栋的私人花园”
\\ \indent 1.------吓人题(76) \indent 2.------吓人题(77)
\\ \indent 3.------吓人题(38)   \indent 4.------吓人题(92)
\\ \indent 5.------吓人题(75)   \indent 6.------吓人题(73)
\\ \indent 7.------吓人题(74)  \indent  8.------吓人题(35)
\\ \indent 9.------吓人题(37)     \indent  10.------吓人题(63)
\\ \indent 11.------吓人题(64)   \indent   12.------吓人题(73)
\\ \indent 13.------吓人题(7)  \indent    14.------吓人题(6)
\\ \indent 15.------吓人题(65)     \indent   16.------吓人题(66)
\\ \indent 17.------吓人题(69)    \indent  18.------吓人题(79)
\\ \indent 19.------吓人题(28)      \indent 20.------吓人题(62)

\part{易错点与空白笔记处}

\chapter{易错点总结}
1. 充分条件与必要条件混淆:“\(p\) 的充分条件是 \(q\)” 等价于 “\(q \implies p\)”,而非 “\(p \implies q\)”。

2. 函数定义域遗漏:求解函数问题(如导数、对数、分式)时,先明确定义域。

3. 互斥与对立关系模糊:对立事件一定互斥,但互斥事件不一定对立。

4. 概念混淆:零点是函数值为零的自变量值(数),极值点是导数为零且左右变号的点(数),\\ \indent\hspace{1em} 均不是坐标点。

5. 错位相减项数错误:相减时注意末项的符号和项数统计。

6. 数列初始条件:求解数列通项或前 \(n\) 项和时,验证 \(n = 1\) 是否满足通项公式。

\chapter{空白笔记处}
(剩余易错点和个人笔记由使用者自行补充)

\end{document}